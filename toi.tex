\documentclass[platex,dvipdfmx]{jlreq}
\usepackage{graphicx}
\usepackage{bm}
\usepackage{url}
\usepackage{tikz}
\usepackage{longtable}
\usepackage{siunitx}
\usepackage{amsmath}
\usepackage{enumerate}
\begin{document}
\section*{問題}
\begin{enumerate}[(1)   ]
  \item $a_1=2\ ,a_{n+1}=2a_n+6$
  \\
  \item $a_1=\ ,a_{n+1}=$
  \\
  \item $a_1=\ ,a_{n+1}=$
  \\
  \item $a_1=\ ,a_{n+1}=$
  \\
  \item $a_1=\ ,a_{n+1}=$
  \\
  \item $a_1=\ ,a_{n+1}=$
  \\
  \item $a_1=\ ,a_{n+1}=$
  \\
  \item $a_1=\ ,a_{n+1}=$
  \\
  \item $a_1=\ ,a_{n+1}=$
  \\
  \item $a_1=\ ,a_{n+1}=$
  \\
  \item $a_1=\ ,a_{n+1}=$
  \\
  \item $a_1=\ ,a_{n+1}=$
  \\
  \item $a_1=\ ,a_{n+1}=$
  \\
  \item $a_1=\ ,a_{n+1}=$
  \\
  \item $a_1=\ ,a_{n+1}=$
  \\
  \item $a_1=\ ,a_{n+1}=$
  \\
  \item $a_1=\ ,a_{n+1}=$
  \\
  \item $a_1=\ ,a_{n+1}=$
  \\
  \item $a_1=\ ,a_{n+1}=$
  \\
  \item $a_1=\ ,a_{n+1}=$
  \\
  \item $a_1=\ ,a_{n+1}=$
  \\
  \item $a_1=\ ,a_{n+1}=$
  \\
  \item $a_1=\ ,a_{n+1}=$
  \\
  \item $a_1=\ ,a_{n+1}=$
  \\
  \item $a_1=\ ,a_{n+1}=$
  \\
  \item $a_1=\ ,a_{n+1}=$
  \\
  \item $a_1=\ ,a_{n+1}=$
  \\
  \item $a_1=\ ,a_{n+1}=$
  \\
  \item $a_1=\ ,a_{n+1}=$
  \\
  \item $a_1=\ ,a_{n+1}=$
  \\
  \item $a_1=\ ,a_{n+1}=$
  \\
  \item $a_1=\ ,a_{n+1}=$
  \\
\end{enumerate}
問題作るときはこういう書き方するんだへー知らなかった.
\end{document}